$Ontext
HoursPerDay : Number of hours operations take place per day
  Source: The report on transportation sector in India shows that the working hours for truck 
  operators range from 8-14 hours. So 10 hours is a reasonable assumption on a average

DieselFuelCost : The cost of diesel in $ per gallon used to run the machines which is used to
  calculate the fuel consumption cost

################  RGY   #####################################

RGYCapitalCostRate : The cost of setting up am RGY in terms of Rs per kg

RGYRentalRate : The rate of storing grains in an RGY on a rental basis in terms of Rs per /
  50 kg bag per month

RGYDryMatterLossRate : The dry matter loss rate for RGY

################  FCI   #####################################

FCIGodownCAPLossRate :  The dry matter loss rate for biomass stored in FCI godowns
  using the covered and plith method

FCIGodownCoveredLossRate :  The dry matter loss rate for biomass stored in FCI godowns using
  the covered (indoor) method

FCIStorageCostRate : The rate of storage of grains in FCI in Rs per kg

FCIStorageCostRateMonthly : The rate of storage of grains in FCI in Rs per kg per month

FCIMandiCostRate : The rate of payment by FCI for different operations at the Mandi in Rs per kg

NumberOfMonthsInFCIStorage : The cost of storage in FCI godowns on a monthly basis

FarmOpenStorageDryMatterLossRate "The dry matter loss rate of grains on an annual basis when
   stored in open on the farm"
  LocalCSPDryMatterLossRate The dry matter loss rate for local CSP facility on an annual basis
  RegionalCSPDryMatterLossRate "The dry matter loss rate for regional CSP facility on an \
    annual basis"
  MillerStorageDryMatterLossRate "The dry matter loss rate for milled material stored in the 
    storage facility of the millers"
  LocalCSPFacilityHeight The height of the facility for local CSP facility in meters /10/
  RegionalCSPFacilityHeight The height of the facility for regional CSP facility in meters /10/
  LocalCSPFacilityAreaLimit "The maximum permitted area for the local CSP facility in square
    meteres" /1000000/
  RegionalCSPFacilityAreaLimit "The maximum permitted area for the regional CSP facility in square
    meteres" /1000000/;

################  Common parameters for storage   #####################################
StorageLandRent "The cost of farm land allocated for storage represented as the land rent in $ 
    per square meter" /0.0395/
* Taken from Wilkerson ASABE 08 presentation
  StorageLandCost "The purchase cost of land to build a centralized storage facility in 
    $ per square mete"r /1.099/
* Taken from the USDA documents on the land cash rent and costs
  BiomassStorageTimeLimit "The number of days after harvesting within which the harvested biomass 
    must be sent to the biorefinery for processing (number of days)" /2/;
* This parameter is used to model Type I energy cane scenario with high sugar content.

*
* Parameters for in-field transportation
*


  StorageStackingCost "Cost of stacking (which includes unloading from the short range bale 
    wagon and stacking at the storage facility) a bale or loaf (at either storage locations) 
    in $ per dry tonne" /1.19/

* Taken from Kumar and Sokhansanj 2007 (IBSAL paper) (value reported for round bale scenario 
* which considers a front end bale loader)
  StorageLoadingCost "Cost of loading the bales or loafs onto truck for long range transportation
   in $ per dry tonne" /1.61/
* Taken from Kumar and Sokhansanj 2007 (IBSAL paper)
  GrainUnloadingCost "Cost of unloading the bales or loafs onto truck for long range transportation
   in $ per dry tonne" /1.46/
* Taken from Kumar and Sokhansanj 2007 (IBSAL paper)
  GroundBiomassStorageLoadingCost "Cost of loading ground biomass onto truck 
    (for ensiling operation) for long range transportation in $ per dry tonne" /3.62/
* Taken from Kumar and Sokhansanj 2007 (IBSAL paper)
          GroundBiomassRefineryUnloadingCost "Cost of unloading ground biomass onto truck (for ensiling operation) for long range transportation in $ per dry tonne" /0.93/
* Taken from Kumar and Sokhansanj 2007 (IBSAL paper)

* Other option for infield transportation cost calculation:
* InFieldTransportRate Infield transportation cost of baled materail in $ per ton /7.34/;
* Taken from Khanna et al. 2008 (originally reported in Duffy and Nanhou)
* Includes cost of renting a tractor, loading it on the field, transporting the bales to the 
* storage area a mile away
* from the field, unloading and stacking and then reloading the truck/semi-trailer for 
* the final hauling to the power plant.

*=========================================================================
* Parameters for transportation
*======================================================================
Parameters
MaximumTransportationFleetSize "Total number of trucks available for all farms 
  for biomass transportation on a day" /100000000000000000000000000/
* Assumed due to lack of data in literature
RailTransportationCost "The operating cost of rail transportation given in 
  Rs per kg per km" /0.001015149/
* Taken from the recent budget modifications (Check the calculations excel sheet)
FCITransportationDryMatterLossRate "Fraction of dry matter lost during transportation 
  related to FCI operations" /0.003267513/
* Check the calculations excel sheet.
TransportationDryMatterLossRate Fraction of dry matter lost during transportation /0.025/
* Taken from kumar & Sokhansanj (2007) which gives loss between 2-3%
MaximumTripsPerDay "The maximum number of trips that a truck can make between different 
  destination in a single day" /10/
* Taken from Mukunda et al. 2006.
TruckLoadingTime "Average loading time for a truck at the farm or the centralized storage location
 in hours" /0.5/
* Taken from Mukunda et al. 2006. They report an average loading time to be 20 minutes for 17 dry 
* tons per truck. The truck considered
* in this model has higher capacity and hence the time is approximated as 30 minutes (0.5 hours). 
* This time includes the actual loading
* time as well as the time spent in the queue for loading (which is expected to be much smaller
*  than at the refinery).
* Taken from kumar & Sokhansanj (2007) which gives loss between 2-3%
TransportationMoistureLossRate Fraction of moisture reduction during transportatation /0.01/
* Assumed due to lack of data in literature
TruckWaitingTime Average waiting time for a truck at the refinery for unloading in hours /0.5/

TruckFuelEfficiency "The average fuel efficiency of a transportation truck of class 7-8 in 
  kilometers per gallon" /10.78/
* Taken from a technical report by Argonne National Lab on heavy vehicles

TruckIdlingFuelConsumption "The typical fuel consumption by transportation truck for class 
  7-8 in gallons per hour" /1/;
* Taken from the presentation given by people from Argonne National Lab


$offtext

################################################################################